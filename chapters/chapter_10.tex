\section{Conclusion} 
At the age of modern technology, we realize the importance of sound classifications. However due to audio data quality sometimes can affect our project in many ways such as by not giving proper output. Our Audio classification can help in many ways for instance audio recommendation, speech recognition etc. So that, we proposed a way to figure out which features can do better performance in an incompatible way. It is also important to focus on the dataset that we are going to use. We wish to find a result that well suits our thesis purpose. It is our belief that if we can find out the sound classifications best fit features, then relevant problems can be discovered with higher focus and efficiency.
\section{Future Work}
In future we will be try to work on Local discriminant bases (LDB)-based audio feature extraction and multigroup classification which are used to find discriminatory time-frequency subspaces. On the basis of the log Mel spectrogram, we also will be work on a convolutional recurrent neural network (CRNN) with learnable gated linear units (GLUs) non-linearity.